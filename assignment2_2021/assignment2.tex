\documentclass{article}
\usepackage[utf8]{inputenc}
\usepackage[margin=1in]{geometry}
\usepackage{amsmath}
\setlength{\parindent}{0em}
\setlength{\parskip}{0.5em}


\title{CTA200 2021 Assignment 2}
\author{DUE: Saturday May 8th by 5:00 PM}
\date{}

\begin{document}

\maketitle

Submit all of your code in a .ipynb file. Please clear all output before commiting your file to Github. You will also submit a .tex file, as described in Question 4. You can see the .tex file for this assignment as a sample of how to get started. I also suggest that those of you new to Latex try out the Overleaf cloud based Latex editor.

\section*{[2] Question 1}

The numerical approximation of the derivative of a mathematical function $f$ evaluated as $x_0$ can be written as:

$$ d_x f|_{x_0} \approx \frac{f_{x_0 + h} - f_{x_0}}{h} $$

Where $h$ is some small ``step''. This is found by looking at the Taylor series of $f(x)$ and taking $h \to 0$. A better approximation, when $h$ is finite (rather than infinitessimal) is:

$$ d_x f|_{x_0} \approx \frac{f_{x_0 + h} - f_{x_0 - h}}{2h} $$

Make a python function that has the form \texttt{def deriv(f, x0, h)} that takes in a python function, and returns the approximation of the derivative at x0 using stepsize h for each of the methods.
Use your functions to take the derivative of the function $f(x) = sin(x)$ for $x_0 = 0.1$ using a variety of values of $h < 1$.
Plot the error compared to the analytical derivative (ie abs(d\_numerical - d\_analytic) / d\_analytic) as a function of $h$ for each method on the same loglog plot.
What do you notice? What does the slope represent?

\section*{[3] Question 2}

For each point in the complex plane $c = x + iy$, with $-2 < x < 2$ and $-2 < y < 2$, set $z_0 = 0$ and iterate the equation $z_{i + 1} = z_i^2 + c$. 
Note what happens to the $z_i$'s: some points will remain bounded in absolute value $|z|^2 = \Re(z)^2 + \Im(z)^2$, while others will run off to infinity. 
Make an image  in which your points $c$ that diverge are given one color and those that stay bounded are given another.
Make a second image where the points are coloured by a colourscale that indicates the iteration number at which the given point diverged.


\section*{[3] Question 3}

The SIR model is a simple mathematical model of disease spread in a population.
The model divides a fixed population of size $N$ into three groups, which vary as a function of time, $t$:

\begin{itemize}
    \item $S(t)$ is those that are susceptible but not yet infected
    \item $I(t)$ is the number of infected individuals
    \item $R(t)$ is those individuals that have recovered and are now immune
\end{itemize}

The model can be described by a set of 3 first order differential equations for each of the variables as

\begin{align}
    \frac{dS}{dt} &= -\frac{\beta S I}{N},\\
    \frac{dI}{dt} &= \frac{\beta S I}{N} - \gamma I,\\
    \frac{dR}{dt} &= \gamma I
\end{align}

Using the ODE integrator of your choice (must be callable in Python, we recommend using Scipy as will be covered in lecture on Friday), integrate the equations with $N=1000$ from $t=0$ to $t=200$ for various values of $\gamma$ and $\beta$ (at least 3-4 values, justify your choices physically).

Use the initial condition $I(0) = 1, S(0)=999, R(0) = 0$ (you can also experiment with other initial conditions if you wish).
Plot the curves for $S, I, R$ on the same figure with a legend. make separate plots for each choice of the parametersi (hint: use the \texttt{subplots()} command).

Bonus (+0.5 mark): Add a 4th parameter $D$ for deaths and justify the addition of a 4th differential equation as well as any RHS terms that must be changed on the initial 3 equations. Integrate the new set of equations for some choice of parameters and comment on the results compared to the SIR model.

\section*{[2] Question 4}

To practice using the \texttt{LaTex}, markup language, we would like you to writeup your results in a .tex file and submit both the \texttt{LaTex} code and the associated PDF. For each of the questions, save and insert your figures into a tex file and write a 1 paragraph methods section, which briefly describes what you did as well as a 1 paragraph analysis section which describes what you see in the results.

\section*{How to Submit}

Submit your assignment by creating a folder called assignment\_2 in your repository from Assignment \#1 and put the notebook, the \texttt{LaTex} file and the PDF output from \texttt{LaTex} as well as any other files which are necessary for me to run your code. Commit the files and push to github. No need to email me your repo again since I have the URL from assignment 1.

\end{document}
